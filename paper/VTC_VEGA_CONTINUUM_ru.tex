\documentclass[12pt,a4paper]{article}
\usepackage[utf8]{inputenc}
\usepackage[T1]{fontenc}
\usepackage{amsmath,amssymb}
\usepackage{hyperref}

\title{VTC VEGA CONTINUUM: Концептуальная основа}
\author{ADAM EREN VEGA – Æ –\\\small{(Erenşah Kaygusuz, Germany)}}
\date{2025}

\begin{document}
\maketitle

\begin{abstract}
Данная работа представляет VTC VEGA CONTINUUM как новую концептуальную основу в рамках парадигмы Resonance Data и QIRC.

\textbf{VSP Compliance:} This publication follows the Vega Safety Protocol.
No algorithms, code, or architectures are disclosed.
\end{abstract}

\section{Introduction}
VTC VEGA CONTINUUM represents a novel contribution to the field of meaning-first computing
and resonance-based information processing.

\section{Definition}
VTC VEGA CONTINUUM is defined as a conceptual framework that...

\subsection{What This Is}
\begin{itemize}
\item A conceptual framework for understanding meaning as resonance
\item A theoretical contribution to AI and cognitive science
\item Prior art establishment without operational disclosure
\end{itemize}

\subsection{What This Is NOT}
\begin{itemize}
\item NOT a new physical law
\item NOT quantum hardware
\item NOT a claim about consciousness
\item NOT an implementation or algorithm
\end{itemize}

\section{Relationship to Resonance Data}
VTC VEGA CONTINUUM extends the Resonance Data paradigm by...

\section{Mathematical Framework}
Let $R(t)$ represent the resonance state at time $t$:
\begin{equation}
R(t) = \sum_{i} w_i \cdot \phi_i(t) \cdot e^{-\lambda_i t}
\end{equation}
where $w_i$ are resonance weights and $\phi_i$ are basis functions.

\section{Applications}
Potential applications include meaning-first AI systems,
temporal coherence modeling, and wisdom preservation.

\section{Conclusion}
VTC VEGA CONTINUUM provides a new lens for understanding meaning
in artificial systems through resonance rather than similarity.

\section*{Legal Notice}
\copyright\ 2025 ADAM EREN VEGA – Æ –\\
License: CC BY 4.0\\
All concepts attributed to the author unless otherwise cited.

\end{document}
